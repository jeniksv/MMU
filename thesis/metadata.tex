%%% Please fill in basic information on your thesis, which will be automatically
%%% inserted at the right places.

% Type of your thesis:
%	"bc" for Bachelor's
%	"mgr" for Master's
%	"phd" for PhD
%	"rig" for rigorosum
\def\ThesisType{bc}

% Language of your study programme:
%	"cs" for Czech
%	"en" for English
\def\StudyLanguage{cs}

% Thesis title in English (exactly as in the official assignment)
% (Note: \xxx is a "ToDo label" which makes the unfilled visible. Remove it.)
\def\ThesisTitle{Multicolor print implementation for 3D printer}

% Author of the thesis (you)
\def\ThesisAuthor{Jan Svojanovsky}

% Year when the thesis is submitted
\def\YearSubmitted{2024}

% Name of the department or institute, where the work was officially assigned
% (according to the Organizational Structure of MFF UK in English,
% see https://www.mff.cuni.cz/en/faculty/organizational-structure,
% or a full name of a department outside MFF)
\def\Department{Department of Software Engineering}

% Is it a department (katedra), or an institute (ústav)?
\def\DeptType{Department}

% Thesis supervisor: name, surname and titles
\def\Supervisor{RNDr. Filip Zavoral, Ph.D.}

% Supervisor's department (again according to Organizational structure of MFF)
\def\SupervisorsDepartment{Department of Software Engineering}

% Study programme (does not apply to rigorosum theses)
\def\StudyProgramme{System Programming}

% An optional dedication: you can thank whomever you wish (your supervisor,
% consultant, who provided you with tea and pizza, etc.)
\def\Dedication{%
{
To my family, thank you for keeping me alive during this thesis writing process and excusing me from household duties.

To my supervisor, RNDr. Filip Zavoral, Ph.D., whose guidance, patience, and vast expertise throughout my entire thesis. Your readiness to review my work, even just days before submission, was immensely appreciated.

To my roommates, thank you for keeping me grounded and bringing laughter into my life with the occasional beer when reality seemed too distant.

A special thanks to Prusa Research for providing the MK4 printer and all the necessary electronics, which were essential to the success of my project.
}
}

% Abstract (recommended length around 80-200 words; this is not a copy of your thesis assignment!)
\def\Abstract{
{
This thesis presents the design, development, and implementation of a Multi-Material Unit (MMU) for enhancing multi-color 3D printing capabilities. We have designed a new MMU prototype, developing its firmware using C++ on a microcontroller platform. The architecture of firmware is modular, facilitating straightforward integration of additional features and functionalities. One of the central part was the development of a communication protocol for interfacing the MMU with a 3D printer, ensuring efficient command and control operations. The system utilizes multiple motors and sensors to manage filament changes. The practicality of the proposed design was validated through a functional proof-of-concept prototype, which demonstrated the MMU's capability to streamline the printing process and reduce system complexity.
}
}
% 3 to 5 keywords (recommended) separated by \sep
% Keywords are useful for indexing and searching for the theses by topic.
\def\ThesisKeywords{%
{firmware developement \sep embeeded system \sep 3D print \sep MMU}
}

% If any of your metadata strings contains TeX macros, you need to provide
% a plain-text version for use in XMP metadata embedded in the output PDF file.
% If you are not sure, check the generated thesis.xmpdata file.
\def\ThesisAuthorXMP{\ThesisAuthor}
\def\ThesisTitleXMP{\ThesisTitle}
\def\ThesisKeywordsXMP{\ThesisKeywords}
\def\AbstractXMP{\Abstract}

% If your abstracts are long and do not fit in the infopage, you can make the
% fonts a bit smaller by this setting. (Also, you should try to compress your abstract more.)
\def\InfoPageFont{}
%\def\InfoPageFont{\small}  % uncomment to decrease font size

% If you are studing in a Czech programme, you also need to provide metadata in Czech:
% (in English programmes, this is not used anywhere)

\def\ThesisTitleCS{Implementace vícebarevného tisku pro 3D tiskárnu}
\def\DepartmentCS{Katedra Softwarového Inženýrství}
\def\DeptTypeCS{Katedra}
\def\SupervisorsDepartmentCS{Katedra Softwarového Inženýrství}
\def\StudyProgrammeCS{Systémové programování}

\def\ThesisKeywordsCS{%
{vývoj firmwaru \sep vestavěný systém \sep 3D tisk \sep MMU}
}

\def\AbstractCS{%
{
Tato práce představuje návrh, vývoj a implementaci jednotky pro více materiálů (MMU), která zlepšuje schopnosti tisku s více barvami ve 3D tisku. Navrhli jsme nový prototyp MMU a vyvinuli jsme jeho firmware v jazyce C++ na platformě mikrokontroléru. Architektura firmware je modulární, což usnadňuje jednoduchou integraci dalších funkcí a možností. Jednou z klíčových částí byl vývoj komunikačního protokolu pro rozhraní MMU s 3D tiskárnou, což zajišťuje efektivní operace příkazů a kontroly. Systém využívá několik motorů a senzorů pro správu změn filamentů. Praktičnost navrhovaného designu byla ověřena prostřednictvím funkčního prototypu, který demonstroval schopnost MMU zjednodušit tiskový proces a snížit složitost systému.
}
}
