\chapter*{Introduction}
\addcontentsline{toc}{chapter}{Introduction}

The integration of Multi-Material Unit (MMU) into Prusa 3D printer MK4 has promised a leap forward in printing capabilities, allowing for the use of multiple materials in a single print job. However, existing solutions, exemplified by the Prusa MMU3, present some practical challenges. This system is cumbersome, expensive, and prone to issues such as system crashes where diagnosing component failures becomes a difficult task. In light of these limitations, this bachelor thesis proposes a proof of concept for a more practical and efficient MMU solution.

The MMU3 enhances 3D printing capabilities by enabling the use of up to five different filaments in a single print job. It incorporates two motors, one for the filament selection and another for the filament pushing. The selector motor navigates a mechanism equipped with a filament sensor, to detect the presence of filament, ensuring the correct filament is aligned with the extruder’s intake path. The motor then engages, driving the sophisticated loading mechanism that feeds the filament into the extruder. This dual-motor setup allows for seamless switching between filaments, facilitating complex multi-material and multi-color printing.
In contrast, our design eliminates the selector motor. Each filament will have its own dedicated motor, and the selector's function will be handled by an innovative mechanical mechanism, streamlining the system and reducing complexity.

Through this thesis, we will explore the design, implementation, and testing of this MMU concept, with a focus on the firmware development process. By presenting a proof of concept for an improved MMU design, I aim to contribute to the advancement of additive manufacturing technology, paving the way for more efficient and user-friendly 3D printing systems.

The thesis is structured to provide a comprehensive understanding of the development and implementation of the MMU. The first chapter offers a technical overview that lays the groundwork by discussing 3D printing technologies. It details the components and processes essential for understanding the challenges in multi-material printing. The second chapter analyzes existing multi-material solutions, highlighting their limitations and setting the stage for the proposed improvements in the MMU. It defines the system requirements needed to enhance the 3D printing capabilities effectively. In the third chapter, the focus shifts to the system components of the MMU, detailing the electronic elements like microcontrollers, motor drivers, and sensors. This section explains the functionality of the Filament Hub, which is central to the proposed improvements. The fourth chapter describes the development environment and tools used in the project, emphasizing the build systems, testing frameworks, and deployment strategies that support the firmware development. The firmware architecture is explored in the fifth chapter, where the firmware design and modular approach are discussed to ensure scalability and integration of future enhancements. Implementation details are provided in the sixth chapter, covering the specifics of protocol management, motor control, sensor integration, and data transmission, crucial for the interaction between the MMU and the printer. The seventh chapter evaluates the performance of the MMU, assessing aspects such as usability, reliability, and efficiency based on testing and user feedback. This evaluation helps identify areas for potential improvement. The thesis concludes by summarizing the significant contributions made towards advancing 3D printing technology and outlining future directions for the research and development of the MMU system. Each chapter builds on the previous ones, progressively detailing the development and refinement of a more efficient MMU for 3D printers.






